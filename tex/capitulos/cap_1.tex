\documentclass[../main.tex]{subfiles}

\begin{document}

\chapter{Derecho de Trabajo}

Busca equilibrar la posición del trabajador y del empleador, dando la protección de vida al trabajador.
Dada la situación asimétrica entre las partes, la forma de corregirlo es con discriminación positiva.

Originalmente, el trabajo era considerado denigrante, el cual era relegado a esclavos y a los más pobres.
E. g. el libro ``El Derecho de los Pobres" es de los primeros de derecho del trabajo.

Con la revolución industrial, grandes multitudes migran hacia las ciudades, donde las condiciones de trabajo eran paupérrimas.
Aquí comienza la explotación de los trabajadores, y como reacción nace la ``cuestión obrera".
Por ejemplo, Ned Ludd propone romper las máquinas para que no remplacen a los obreros, inspirando a los luditas.

Junto a la maquinería nace el modelo de manufactura, por lo que se vuelve más importante controlar el comercio que la materia prima.
El área de las finanzas se desarrolla y se vuelve más valioso el control del capital que el control de la tierra.

En el siglo XVII ocurren las revoluciones francesa y americana.
Surgen conceptos como los derechos humanos, la igualdad ante la Ley, etc., que devienen en el concepto moderno del estado garantista.
Sin embargo faltará esperar hasta que la ``cuestión social" mejore.
En esta época la miseria era el común denominador, por lo que surgen reacciones como el socialismo, comunismo, etc.

El trabajador generalmente \textbf{trabaja por necesidad,} lo cual genera un desequilibrio con el empleador, que puede o no darle empleo.
Dado este desequilibrio, se vuelve dificil la aplicación correcta y equitativa del derecho civil.
Por esto nace el derecho de trabajo.
El trabajador no puede negociar cláusulas de contratos, ya que si pide mucho hay un ``ejército de reserva"\footnote{Expresión utilizada por Marx.} listo para sustituirlo.
Además, el trabajador pone a disposición del empleador su tiempo, que es lo más valioso que tiene.

La primer cuestión que unió a los trabajadores fue la jornada de 8 horas (sancionada en 1915 en Uruguay).
El derecho del trabajo, entonces, está hecho para \textbf{proteger a los trabajadores,} no para el empleador.
Este último lo debe cumplir, y generalmente quiere que las normas sean claras.

En el comercio la jornada está limitada a 44h semanales, mientras que en la industria a 48h.
Luego se agregó el sector de servicios, limitado a ??h.

Debajo de la relación patrón-obrero subyace el conflicto, que si bien no es algo negativo, existe.
Conviene resolverlo por vía del diálogo, de manera tal que no estalle y cause problemas al resto de la sociedad.
En estos casos, el gobierno intercede para resolverlo, en línea con el interés general.
El derecho al trabajo \textbf{no busca limitar el conflicto,} sino enmarcarlo en cierta normativa.

¿Cómo se dicta el derecho del trabajo?
De forma heterónoma (e.g. el Estado) y de forma autónoma (empleador y trabajadores).
Además, en el derecho del trabajo \textbf{la costumbre genera derecho.}
Si algo siempre se hace así, el empleador no puede cambiarlo sin anunciarlo previamente.

Las negociaciones entre las partes (sindicatos, estado, empleadores) están marcadas por las presiones típicamente sindicales, como paros, huelgas, etc.
De la misma forma hay cláusulas que buscan resolver conflictos antes de necesitar de estas presiones.

Además de los motivos expresados antes, el DdT también sirve un propósito económico, un propósito de redistribución de la riqueza, y un propósito de justicia social.

Existe una diferencia entre el derecho individual y el derecho colectivo del trabajo.

\section{Derecho colectivo}

Los trabajadores se dieron cuenta que juntos tenían mucho más poder de negociación, por lo que formaron sindicatos y gremios.
Esto hace que el DT sea particularmente distinto del derecho civil.

En Uruguay está establecida la libertad sindical (derecho a afiliarse o no a un sindicato, etc.).

\section{Grado de responsabilidad o renuncia}

En otras áreas, uno puede alquilar una casa de 20k por 10k y el Estado no va a intervenir.
Sin embargo, el DT es una norma de orden público, por lo que no puede ser renunciado incluso por particulares.
Uno no puede rechazar un derecho o beneficio que la Ley le otorga.

\section{Derechos fundamentales}

Hay algunos derechos de tal importancia que se requiere una acción societal (normalmente del Estado) para que todos pudan ejercerlos (e.g. la vivienda, la salud).

En el DT, uno de ellos es el derecho al trabajo.
El artículo 38 de la Constitución lo establece en Uruguay.
Esto no significa que cualquiera pueda tener cualquier trabajo, sino que todos tenemos derecho a trabajar.
Por ejemplo, uno puede discriminar por habilidad o formación, pero no puede decir "solo contrato varones cisgénero heterosexuales".

En Uruguay hay aspectos del derecho del trabajo que están constitucionalizados, así como en otros países.
Al estar en la Constitución, quedan en la cúspide de la jerarquía legal, lo cual le da preeminencia por sobre otras leyes de nivel menor, como pueden ser los contratos particulares.

\section{DT y economía}

El DT está fuertemente ligado a la economía.
Algunas corrientes más liberales sugieren que la regulación impide el correcto funcionamiento de la economía, mientras que otras proponen que cuanto más regulación, mejor para la sociedad.

\section{Otras características del DT}

\begin{itemize}

\item \textbf{Fragmentario:} El DT no está codificado en un solo material, sino que está desperdigado en varias leyes, decretos, etc.

\item \textbf{Internacionalidad:} La OIT establece ciertas normas básicas que sus estados miembros deben cumplir

\item \textbf{Protectividad:} Es una particularidad del DT, ya que no es un punto de partida sino uno de llegada. Su propósito es la protección del trabajo.

\item \textbf{Subyacencia del conflicto}

\item \textbf{Colectividad:} Cuando se establecen normas de DT, se debe negociar siempre con las partes más representativas.

\item \textbf{Significación del tiempo social:} El DT está ligado a las sensibilidades del momento actual. E.g. hoy en día no se considera el trabajo como algo masculino exclusivamente.

\item \textbf{Autonomía:} ``Científica" (resuelve sin necesidad de otras ramas del derecho), legislativa, etc.

\end{itemize}

\section{Personería de las partes}

El hecho de volverse empleador conlleva ciertas responsabilidades para con el Estado y los trabajadores.
Su incumplimiento o evasión trae consigo ciertas penas, sin importar si hay sociedades anónimas o alguna entidad de por medio.

En Uruguay, los sindicatos tienen un estado jurídico nebuloso, y todavía no está reglamentado si son personas jurídicas o no.
\footnote{NE: A la fecha (7 de julio de 2022) hay discusiones políticas que tal vez resuelvan esto.}

En el Derecho Laboral (DL) no se anulan procesos por errores menores, como pueden ser errores ortográficos, errores de designación, etc.
Los trabajadores deben poder canalizar sus conflictos; no pueden quedar boyando sin saber cómo actuar.
Es responsabilidad de la sociedad asesorar al empleado para que no deje pasar injusticias que lo perjuician.
En pos de esto, los juicios de DL \textbf{no} requieren el pago de un timbre profesional, por ejemplo.

\subsection{Conjunto económico}

Conjunto de empresas que son formalmente diferentes pero tienen ciertas características que las unen (e.g. mismos dueños).
En caso de demandas o conflictos la justicia puede ir contra todo el conjunto en lugar de solamente apuntar contra el elemento en falta.

Pla define un conjunto económico como: \blockquote{un núcleo de empresas aparentemente autónomas pero sometidas a una dirección económica única}

Pueden existir por descomposición (en sub-empresas, etc.) o por concentración de unidades preexistentes.

Muchas veces no parece existir una relación cuando uno mira desde el exterior, pero en lo administrativo/organizativo hay una relación íntima entre partes.

\section{Subcontratación / Tercerización}

\section{Principios del DT}

\section{Particularismos}

\end{document}