\documentclass[../main.tex]{subfiles}

\begin{document}

\chapter{Derecho de Trabajo}

Busca equilibrar la posición del trabajador y del empleador, dando la protección de vida al trabajador.
Dada la situación asimétrica entre las partes, la forma de corregirlo es con discriminación positiva.

Originalmente, el trabajo era considerado denigrante, el cual era relegado a esclavos y a los más pobres.
E. g. el libro ``El Derecho de los Pobres" es de los primeros de derecho del trabajo.

Con la revolución industrial, grandes multitudes migran hacia las ciudades, donde las condiciones de trabajo eran paupérrimas.
Aquí comienza la explotación de los trabajadores, y como reacción nace la ``cuestión obrera".
Por ejemplo, Ned Ludd propone romper las máquinas para que no remplacen a los obreros, inspirando a los luditas.

Junto a la maquinería nace el modelo de manufactura, por lo que se vuelve más importante controlar el comercio que la materia prima.
El área de las finanzas se desarrolla y se vuelve más valioso el control del capital que el control de la tierra.

En el siglo XVII ocurren las revoluciones francesa y americana.
Surgen conceptos como los derechos humanos, la igualdad ante la Ley, etc., que devienen en el concepto moderno del estado garantista.
Sin embargo faltará esperar hasta que la ``cuestión social" mejore.
En esta época la miseria era el común denominador, por lo que surgen reacciones como el socialismo, comunismo, etc.

El trabajador generalmente \textbf{trabaja por necesidad,} lo cual genera un desequilibrio con el empleador, que puede o no darle empleo.
Dado este desequilibrio, se vuelve dificil la aplicación correcta y equitativa del derecho civil.
Por esto nace el derecho de trabajo.
El trabajador no puede negociar cláusulas de contratos, ya que si pide mucho hay un ``ejército de reserva"\footnote{Expresión utilizada por Marx.} listo para sustituirlo.
Además, el trabajador pone a disposición del empleador su tiempo, que es lo más valioso que tiene.

La primer cuestión que unió a los trabajadores fue la jornada de 8 horas (sancionada en 1915 en Uruguay).
El derecho del trabajo, entonces, está hecho para \textbf{proteger a los trabajadores,} no para el empleador.
Este último lo debe cumplir, y generalmente quiere que las normas sean claras.

En el comercio la jornada está limitada a 44h semanales, mientras que en la industria a 48h.
Luego se agregó el sector de servicios, limitado a ??h.

Debajo de la relación patrón-obrero subyace el conflicto, que si bien no es algo negativo, existe.
Conviene resolverlo por vía del diálogo, de manera tal que no estalle y cause problemas al resto de la sociedad.
En estos casos, el gobierno intercede para resolverlo, en línea con el interés general.
El derecho al trabajo \textbf{no busca limitar el conflicto,} sino enmarcarlo en cierta normativa.

¿Cómo se dicta el derecho del trabajo?
De forma heterónoma (e.g. el Estado) y de forma autónoma (empleador y trabajadores).
Además, en el derecho del trabajo \textbf{la costumbre genera derecho.}
Si algo siempre se hace así, el empleador no puede cambiarlo sin anunciarlo previamente.

Las negociaciones entre las partes (sindicatos, estado, empleadores) están marcadas por las presiones típicamente sindicales, como paros, huelgas, etc.
De la misma forma hay cláusulas que buscan resolver conflictos antes de necesitar de estas presiones.

Además de los motivos expresados antes, el DdT también sirve un propósito económico, un propósito de redistribución de la riqueza, y un propósito de justicia social.

Existe una diferencia entre el derecho individual y el derecho colectivo del trabajo.

\section{Derecho colectivo}

Los trabajadores se dieron cuenta que juntos tenían mucho más poder de negociación, por lo que formaron sindicatos y gremios.
Esto hace que el DT sea particularmente distinto del derecho civil.

En Uruguay está establecida la libertad sindical (derecho a afiliarse o no a un sindicato, etc.).

\section{Grado de responsabilidad o renuncia}

\section{Derechos fundamentales}

\section{DT y economía}

\section{Otras características del DT}

\section{Personería de las partes}

\section{Subcontratación / Tercerización}

\section{Principios del DT}

\section{Particularismos}

\end{document}